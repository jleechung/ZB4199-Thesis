
%%%%%%%%%%%%%%%%%%%%%%%%%%%%%%%%%
%% Requirements 
%%%%%%%%%%%%%%%%%%%%%%%%%%%%%%%%%

% (a) Organisation and Presentation of Thesis (i.e. clarity and logical sequence)
% (b) Coverage (i.e. including literature)
% (c) Depth and Quality of research, originality and critical ability shown (i.e scientific hypothesis,
% computational method and validation design, if the conclusion is justifiable by results obtained)

%%%%%%%%%%%%%%%%%%%%%%%%%%%%%%%%%
%% Packages 
%%%%%%%%%%%%%%%%%%%%%%%%%%%%%%%%%

\documentclass[12pt, twoside]{report}
\usepackage[utf8]{inputenc}
\usepackage[a4paper, left=22mm, right=22mm, top=30mm, bottom=25mm]{geometry}
\usepackage{lipsum}
\setlength{\headheight}{15pt}
\linespread{1.25}
\raggedbottom
\usepackage{float}

%% Header and footer formatting
\usepackage{fancyhdr}
\pagestyle{fancy}
\fancyhf{}
\fancyhead[R]{\thepage}
\renewcommand{\chaptermark}[1]{\markboth{#1}{}}
\fancyhead[L]{\chaptername~\thechapter.~\leftmark}
\renewcommand{\headrulewidth}{0pt}

%% TOC
\usepackage[nottoc]{tocbibind}

%% Chapters
\usepackage{titlesec}
\titleformat{\chapter}[display]
{\LARGE\bfseries}{\chaptertitlename~\thechapter}{10pt}{\LARGE}
\titlespacing*{\chapter}{0pt}{-40pt}{10pt}
\titlespacing*{name=\chapter,numberless}{0pt}{-40pt}{10pt}

%% Glossaries
\usepackage[toc, nonumberlist]{glossaries}
\newglossarystyle{modsuper}{%
  \glossarystyle{super}%
  \renewcommand{\glsgroupskip}{}%
  \renewcommand*{\glossaryentryfield}[5]{%
    \glsentryitem{##1}\glstarget{##1}{##2} & ##3\glspostdescription\space ##5\\[5pt]}%
}
\setglossarystyle{modsuper}
\makeglossaries
\loadglsentries{glossaries}
\glsaddall

%% References
\usepackage[sorting=none, style=numeric-comp]{biblatex}
\addbibresource{references.bib}

%% Figures
\usepackage{graphicx}
\graphicspath{{figures/}}
\usepackage{caption}
\usepackage{subcaption}

%% Hyperlinks
\usepackage{hyperref}

%% Maths
\usepackage{amsfonts}
\usepackage{amsmath}
\usepackage{amsthm}
\theoremstyle{definition}
\newtheorem{definition}{Definition}[section]
\usepackage{bm}
\newcommand{\ts}{\textsuperscript}
\newcommand{\irow}[1]{% inline row vector
  \begin{smallmatrix}(#1)\end{smallmatrix}%
}
\DeclareMathOperator*{\argmax}{arg\,max}

%% Graphical models
\usepackage{tikz}
\usetikzlibrary{bayesnet}

%% Line numbers
\usepackage[switch, modulo]{lineno}

%% Font
\renewcommand{\familydefault}{\sfdefault} 
\usepackage{caption}
\captionsetup{font=footnotesize}

%% Tables
\usepackage{array}
\newcolumntype{P}[1]{>{\centering\arraybackslash}p{#1}}
\usepackage{bigstrut}

\theoremstyle{definition}
\newtheorem{exmp}{Example}[section]

%% Code
\usepackage{listings}

%% txt include
\usepackage{fancyvrb}
% redefine \VerbatimInput
\RecustomVerbatimCommand{\VerbatimInput}{VerbatimInput}%
{fontsize=\footnotesize,
 %
 frame=lines,  % top and bottom rule only
 framesep=2em, % separation between frame and text
 rulecolor=\color{Gray},
 %
 label=\fbox{\color{Black}data.txt},
 labelposition=topline,
 %
 commandchars=\|\(\), % escape character and argument delimiters for
                      % commands within the verbatim
 commentchar=*        % comment character
}

%%%%%%%%%%%%%%%%%%%%%%%%%%%%%%%%%
%% Document starts here
%%%%%%%%%%%%%%%%%%%%%%%%%%%%%%%%%

\begin{document}

\pagenumbering{gobble}


\begin{titlepage}
   \begin{center}
       \vspace*{1cm}
       
       \Large
       \textbf{Statistical models for transcript quantification from long-read RNA-seq}
       
       \vspace{1.5cm}
       
       \large
       by\\
       
       \vspace{0.25cm}
       
       Joseph Lee Jing Xian
       
       \vspace{1.5cm}
       
       Honours Project in Computational Biology
       
       \vspace{0.25cm}
       
       Faculty of Science
       
       \vspace{0.25cm}
       
       National University of Singapore
       
       \vspace{1.5cm}
       
       Supervisor:
       
       \vspace{0.25cm}
       
       Dr. Jonathan Göke
       
       \vspace{0.25cm}
       
       Genome Institute of Singapore
       
       \vspace{1.5cm}
       
       Co-supervisor:
       
       \vspace{0.25cm}
       
       Prof. Greg Tucker-Kellogg
       
       \vspace{0.25cm}
       
       National University of Singapore
       
       \vfill
       
       May 2022

   \end{center}
\end{titlepage}

\chapter*{Abstract}

The development of long-read RNA-seq technologies has enabled the profiling of full-length reads while mitigating biases in previous generations of RNA-seq technologies, improving the accuracy of isoform abundance estimates. However, biases present in long-read RNA-seq data and their effects on transcript quantification have not yet been extensively explored in the literature. 

In this thesis, we examine \textit{degradation bias} present in long-read direct RNA-seq, where reads are truncated and map to multiple isoforms, leading to ambiguity in read-to-transcript assignment and erroneous isoform abundance estimates. We characterise degradation in real datasets, develop a bias-aware model for transcript quantification and develop expectation maximization and variational Bayesian algorithms for inferring isoform abundance estimates. By accounting for degradation bias, we demonstrate improvements in transcript quantification on simulated datasets with known degradation rate and real datasets with sequencing spike-ins.



\chapter*{Acknowledgements}

I thank Jonathan Göke for conceptualising this work and granting me the opportunity to work on it. I also thank him for his continual guidance, not only for the duration of this thesis, but also from the time I joined his lab in the summer of 2020. His support and patience have been invaluable.\\[10pt]
My gratitude extends also to Chen Ying and Andre Sim for their contributions to this work: the weekly discussions on this subject, the resources they have provided, and the suggestions they have given me have all shaped this work crucially. Their own work is brilliant and has given me much insight and piqued my interest in all things long-read.\\[10pt]
I also thank Prof. Greg Tucker-Kellogg for his co-supervision and for conducting great courses in bioinformatics at NUS that have laid foundations instrumental to this work.\\[10pt] 
Last but not least, I thank Prof. Choi Hyungwon for playing a large part in my statistical education which has come to fruition with this work.  

\tableofcontents

\pagenumbering{roman}

\listoffigures

\listoftables

\printglossaries

\chapter{Introduction}
\pagenumbering{arabic}

Third generation sequencing technologies have enabled the production of long reads ranging from tens to hundreds of kilobases in length \cite{Pollard2018}, and have shown promise in resolving many challenges in genomics and transcriptomics \cite{Bolisetty2015, Byrne2017, DeCoster2019, Liu2019, Mantere2019, Nurk2021}. In particular, long-read technologies enable greater insight into the transcriptome and its complexity, which is crucial in understanding the functioning of cells and their biological processes. These technologies allow accurate detection of full-length transcripts and novel splice junctions while mitigating biases associated with short-read technologies, enabling more accurate quantification of reference and novel isoforms. Nevertheless, biases are still present in long-read technologies, albeit to a lesser extent. 

This project focuses on a particular bias present in long-read \gls{rnaseq} referred to as \textit{degradation bias}. This bias arises due to the fact that from the time a transcript is synthesized to when it is sequenced, it is subject to multiple factors that results in its degradation, primarily from the 5' end of the transcript. Consequently, the observed reads are often truncated and \textit{degraded}, potentially resulting in ambiguity in read assignment to transcript isoforms. This, in turn, leads to erroneous quantification estimates. As degradation bias in long-read \gls{rnaseq} has not been extensively covered in existing literature, we attempt to do so here by characterising the bias and its effects on quantification from long-read \gls{rnaseq}, and developing a framework to model and correct such bias by producing \textit{degradation-aware quantification estimates}.

\section{Review}

Here, we review various concepts and existing literature relevant to our aim of modeling bias in long-read \gls{rnaseq}. We first examine (i) biases in short-read \gls{rnaseq} technologies and how they are accounted for by existing methods, which provide some ideas on how to handle biases in long-read \gls{rnaseq}. Next, we review (ii) long-read \gls{rnaseq} technologies and how they mitigate biases in (i), and (iii) biases that long-read technologies themselves possess. 



\chapter{Characterising degradation}

In this chapter, we aim to characterise transcript degradation and read truncation from long-read RNA-seq data. Transcript degradation from the 5' end results in truncated reads and a decrease in coverage with increasing distance from the 3' end for a given isoform. Thus, even though the observed degradation occurs from the 5' end, it is helpful to characterise degradation as the resultant decrease in coverage from the 3' end. We formalize the notion of degradation by defining the \textit{degradation rate}. 
\begin{definition}[Normalized coverage]
Let the maximum coverage over an isoform be $\mathrm{cov}_{\mathrm{max}}$ and the coverage at base $b$ be $\mathrm{cov}_b$. The normalized coverage of the isoform at base $b$ is defined as 
\begin{equation}
    \mathrm{ncov}_b=\frac{\mathrm{cov}_b}{\mathrm{cov}_{\mathrm{max}}}
\end{equation}
\end{definition}
\begin{definition}[Degradation rate]
Let $\mathrm{ncov}$ be the normalized coverage over an isoform, and $x$ be the distance in kb from the 3' end. The degradation rate of an isoform is defined as the rate of change in normalized coverage with respect to distance in kb from the 3' end of the isoform:
\begin{equation}
    d=\lim_{\Delta x\rightarrow 0} \frac{\Delta \mathrm{ncov}}{\Delta x}
\end{equation}
The degradation rate of an isoform at a base $b$ with distance $x_b$ away from the 3' end is the value of this limit evaluated at $x=x_b$. 
\end{definition}
Intuitively, the degradation rate can be interpreted as the gradient of the plot of normalized coverage against distance from the 3' end. We will refer to this plot as the \textbf{degradation curve}. To illustrate this, we visualise degradation curves for a hypothetical isoform of length 2 kb with different degradation rates (Fig. \ref{fig:sec-2-hypo}).  
\begin{figure}[H]
    \centering
    \includegraphics[width=\textwidth]{figures/sec-2-hypo.png}
    \caption[Normalized coverage plots for a hypothetical isoform]{Normalized coverage plots for a hypothetical isoform of length 2 kb illustrating different degradation rates. \textbf{a.} Degradation rate is 0. \textbf{b.} Degradation rate is constant (0.2). \textbf{c.} Degradation rate is variable.}
    \label{fig:sec-2-hypo}
\end{figure}
In Fig. \ref{fig:sec-2-hypo}a, the degradation rate or gradient is 0, implying that all reads from the isoform are full-length, with no drop in coverage over the isoform body. Conversely, in Fig. \ref{fig:sec-2-hypo}b, the gradient is a constant value of 0.2 over the isoform body, implying that for every 1 kb from the 3' end, normalized coverage drops by 0.2. The last plot in Fig. \ref{fig:sec-2-hypo}c shows variable gradient over the isoform body. In particular, the gradient is low in magnitude towards the 3' end and increases with distance from the 3' end.  

In the following sections, we first describe a coverage-based approach for characterising degradation in long-read direct RNA-seq data, and validate this approach with simulated data where the degradation is known. We then examine degradation in real data, characterising degradation by isoform features and in sequencing spike-ins. Finally, we develop a method for efficient read length-based degradation estimation that can be used for obtaining degradation-aware isoform abundance estimates.

\section{Coverage-based degradation estimation}\label{sec:deg-est}

We first sought to determine if patterns of degradation were consistent across transcript isoforms for a given direct RNA-seq dataset. To that end, we selected single-isoform, multi-exon genes from GRCh38 reference annotations, and further restricted the set of isoforms to those that do not intersect with any other annotated features in reference annotations. We refer to these as \textbf{lone isoforms}. Estimating bias from lone isoforms reduces ambiguity in the isoform of origin of the reads we use to estimate bias; such approaches were also adopted in \cite{Roberts2011} and \cite{Love2016} for bias estimation in short-read data. Filtering yielded approximately 5,000 lone isoforms. For each of these isoforms, we obtained coverage over the isoform body using the \texttt{genomecov} module from bedtools (v.2.27.1). Next, we further apply a median coverage filter, filtering out lone isoforms with low median coverage (min median coverage = 10). Finally, a degradation curve is fitted to the data with a smoothing spline.  

To validate this approach, we simulated two datasets where the expected degradation rates for all isoforms is constant ($\mathbb{E}[d]\in\{0.2,0.4\}$, see Appendix \ref{ap:sim-deg-reads} for more details on degraded read simulation). Visualising the degradation curves for each isoform, we observed noise in the form of deviation from the expected degradation curve, likely due to sampling noise in the simulation data generation process. This noise can be mitigated by computing a global degradation curve across all isoforms (red line, Fig. \ref{fig:cov-sim}). We find that the global degradation curves reflect the known degradation rates from the simulated data qualitatively (Fig. \ref{fig:cov-sim}). In addition, we assessed the global degradation curves quantitatively by computing the average gradients of each curve, and found good agreement between the estimates and the expected degradation ($\mathbb{E}[d]$ = 0.2, estimated = 0.201; $\mathbb{E}[d]$ = 0.4, estimated = 0.403).

We applied this approach in real direct RNA-seq data from the SG-NEx project. Within each sample, we found consistent patterns of degradation across isoforms (Fig. \ref{fig:cov-real}). However, we note that for most real datasets, filtering for lone transcripts and by median coverage yields very few isoforms remaining for estimating degradation rates. For instance, in a HepG2 sample (Fig. \ref{fig:cov-real}a), we obtained only 57 isoforms, while in a MCF7 sample (Fig. \ref{fig:cov-real}b), we obtained only 52 isoforms. Across the samples, the median number of lone isoforms post-filtering was 21. While this approach separates signal and noise by considering only lone isoforms, it may be overly restrictive. We consider an improved approach for estimating the degradation rate in Section \ref{sec:rld} based on observed read length distributions. 

\begin{figure}[H]
    \centering
    \includegraphics[width=0.9\textwidth]{figures/sec-2-cov-sim.png}
    \caption[Degradation curves on simulation datasets based on coverage]{Degradation curves on simulation datasets based on coverage. \textbf{a.} Degradation curve for dataset with $\mathbb{E}[d]$ = 0.2. Coverage drops close to 0 at 5kb with a degradation rate of 0.2. The estimated degradation rate is 0.201. \textbf{a.} Degradation curve for dataset with $\mathbb{E}[d]$ = 0.4. Coverage drops close to 0 at 2.5 kb. The estimated degradation rate is 0.403.}
    \label{fig:cov-sim}
\end{figure}

\begin{figure}[H]
    \centering
    \includegraphics[width=0.9\textwidth]{figures/sec-2-cov-real.png}
    \caption[Degradation curves on real datasets based on coverage]{Degradation curves on real datasets based on coverage. These two datasets has the most isoforms post-filtering for degradation rate estimation. \textbf{a.} Degradation curve for a HepG2 cell line sample. \textbf{b.} Degradation curve for a MCF7 cell line sample.}
    \label{fig:cov-real}
\end{figure}

\subsection{Degradation by isoform features}

Here, we explore whether degradation rates vary between isoforms of different features. In particular, We striate isoforms by their annotated length , observed median coverage (Fig. \ref{fig:cov-feature}b) and biotype (Fig. \ref{fig:cov-feature}c). On a representative sample from the MCF7 cell line, estimated degradation rates do not appear to vary significantly between isoforms of different annotated length or median coverage (Fig. \ref{fig:cov-feature}a,b). However, the converse is true for different transcript biotypes. In particular, we observe higher degradation rates for processed pseudogenes (n = 7) and long non-coding RNAs (n = 3) as compared to protein coding genes (n = 42, Fig. \ref{fig:cov-feature}c). While the sample size here is small, this result corroborates findings in \cite{Clark2012} and \cite{Kaiwan2021}, which identified larger variance in the stabilities of lncRNAs.   

\begin{figure}[H]
    \centering
    \includegraphics[width=0.95\textwidth]{figures/sec-2-by-feature.png}
    \caption[Degradation curves in MCF7 striated by features]{Degradation curves in MCF7 striated by features. \textbf{a.} Degradation curves fit for isoforms within each length quantile separately, with quantiles at 0\%: 320, 25\%: 1360, 50\%: 2820, 75\%: 5590, 100\%: 9038. \textbf{b.} Degradation curves fit for isoforms within each median coverage quantile separately, with quantiles at 0\%: 7, 25\%: 12, 50\%: 18, 75\%: 31, 100\%: 381. \textbf{c.} Degradation curves fit for isoforms of different biotypes.}
    \label{fig:cov-feature}
\end{figure}

\subsection{Degradation in spike-ins}

As reviewed in the first chapter, the observed read truncation in ONT datasets may be a result of RNA degradation \textit{in vivo} or due to other extra-cellular factors including library preparation or sequencing artifacts. In this section, we analyse possible degradation in sequencing spike-ins, which are synthetic RNA molecules added to endogenous RNA samples before library preparation \cite{Lexogen20201}. By doing so, we attempt to elucidate the relative contributions of \textit{in vivo} RNA decay and other extraneous factors unrelated to decay.

We analyze SIRVs (Spike-in RNA Variants) present in a subset of SG-NEx samples. Fortunately, a subset of SIRV isoforms are non-overlapping, allowing us to easily apply coverage-based approaches for estimating the degradation. The degradation curves estimated on the SIRVs for six H9 samples show consistent patterns both amongst the samples and with degradation rates  suggesting that extraneous factor such as library preparation or sequencing do play a part in the observed degradation and read truncation.    

\begin{figure}[H]
    \centering
    \includegraphics[width=0.9\textwidth]{figures/sec-2-cov-spike.png}
    \caption[Degradation curves for SIRVs based on coverage]{Degradation curves for SIRVs based on coverage for six H9 samples. The number of SIRV transcripts used to fit the degradation curve is shown in parentheses.}
    \label{fig:cov-spike}
\end{figure}


\section{Read length-based degradation estimation}\label{sec:rld}



\chapter{Bias-aware quantification}

\lipsum[5]

\section{Bias-aware model for quantification}

\lipsum[5]

\section{Parameter inference}

\lipsum[5]




\chapter{Model evaluation and results}

In this chapter, we evaluate our quantification model and inference algorithms on simulated datasets with constant expected degradation rates and real direct RNA-seq datasets with sequencing spike-ins. 

\section{Read alignment}

Simulated and real reads were aligned with minimap2 \cite{Minimap2018, Minimap2021}, a popular aligner for long reads. For alignment to the genome, we used the flags \texttt{-ax splice -uf -k14} that take into account splicing and forward strand alignment for ONT direct RNA-seq as recommended by minimap2 developers. For alignment to the transcriptome, we use default flags \texttt{-ax map-ont -N10} for mapping ONT reads, keeping 10 secondary alignments due to similarities in transcript isoform sequences. We used the same genomic or transcriptomic alignments across the tools, depending on which is required as input.   

\section{Model variations}

We compare two variations of our model based on the read length-isoform agreement:

\paragraph{Deg. EM (exact)} We ran our model with the exact read length-isoform agreement model (Eqn. \ref{eq:read-iso-agreement}) and provided the known degradation rate to the model for inference. This is of course unrealistic, since in real data, the degradation rate is not known \textit{a prior} and there is no bound on the maximum read length (i.e., the degradation rate is not constant). Nevertheless, this serves as a useful sanity check that our approach works. 

\paragraph{Deg. EM (emp.)} The second variation of our model uses the empirical read length-isoform agreement model (Eqn. ) and estimates the degradation rate per base from the data. This variation has no limitation on the maximum read length nor constraints on the degradation rate. 

\section{Methods for benchmarking}

We benchmark our model against three existing methods for transcript quantification from long-read RNA-seq. For brevity, detailed descriptions of these methods are omitted here but included in Appendix \ref{ap:meth}.

\paragraph{Bambu} Bambu (manuscript in review) \cite{Bambu2022} is a method for reference guided transcript discovery and quantification for long read RNA-seq data. Crucially, Bambu is one of the few existing long-read quantification methods that models degradation bias. For benchmarking, we ran Bambu (v.2.0.6) without transcript discovery. For transcript quantification, we ran Bambu with bias correction (default) and without bias correction (\texttt{degradationBias=FALSE}). We used the defaults for all other parameters.

\paragraph{FLAIR} Full-Length Alternative Isoform analysis of RNA (FLAIR) \cite{Tang2020} is a method for correction, isoform definition and alternative splicing analysis of long reads. For benchmarking we ran FLAIR (v.1.5.1) with the modules \texttt{correct}, \texttt{collapse} and \texttt{quantify}. When running FLAIR on simulated data, we set \texttt{--support 1} for FLAIR \texttt{collapse}, keeping isoforms that are supported by minimally one read (default=3).  

\paragraph{NanoCount} NanoCount \cite{Gleeson2021} is a method for quantifying isoform abundance from ONT direct RNA-seq data. Of the three methods listed here, it is the most comparable to our model as it is tailored for direct RNA-seq and uses an expectation maximization algorithm for estimating isoform abundance estimates. For benchmarking, we ran NanoCount (v.1.0.0.post6) with default parameters.

\section{Evaluations on simulated data}\label{sec:eval-sim}

To evaluate our model's ability to correct for degradation bias, we simulated five datasets for a range of degradation rates $\mathbb{E}[d]\in\{0.05,0.1,0.2,0.4,0.5\}$ (Appendix \ref{ap:sim-deg-reads}). In addition, we simulated reads for artificial novel isoforms that are modified by dropping exons from the 5' end of selected reference isoforms, termed \textit{subset} isoforms (Appendix \ref{ap:gen-novl-iso}, Fig. \ref{fig:app-a-1}). This increases the proportion of multi-mapping reads and makes correcting for degradation bias crucial for accurate transcript quantification. 

The read counts for simulation follow a negative binomial distribution, which is often used for modeling RNA-seq counts and other count data that is over-dispersed, i.e., where the assumption of equal mean and variance is not held \cite{Cameron2013, Anders2010, Robinson2010}. We parameterise the distribution with the mean $\mu$ and a dispersion parameter $\alpha$ such that the variance is given by $\mu+\alpha\mu^2$. This is the same parameterisation used in \cite{Robinson2010}. To ensure that the negative binomial is a valid choice of distribution, we fit discrete distributions to the counts returned by existing methods on real long-read RNA-seq data, and find that the negative binomial provides a better fit to the data compared to the Poisson distribution (Appendix \ref{ap:count-dist}).   

\subsection{Comparisons between model variations}

In this section, we compare the Deg. EM (exact) and Deg. EM (emp.) models based on the isoform abundance estimates obtained. We compare these estimates against the simulated ground truth based on Spearman correlation (SCC), normalized root-mean-squared error (NRMSE) and median relative difference (MRD). Explanations of these metrics can be found in Appendix \ref{ap:eval-metrics}.  

Both variations of the model perform comparably well on the simulated data, achieving SCC $>$ 0.7 across all five simulated datasets with low NRMSE and MRD. Table \ref{tab:summary-1} shows the mean of each metric across the five datasets for all and subset isoforms. We first note that performance on the subset isoforms is poorer compared to that on all isoforms, for both variations and across metrics. This is expected as there is more ambiguity in read assignment with the subset isoforms. 

% Table generated by Excel2LaTeX from sheet 'sec-4-1-table'
\begin{table}[htbp]
\centering
  \resizebox{\columnwidth}{!}{%
\begin{tabular}{|l|P{2cm}|P{2cm}|P{2cm}|P{2cm}|P{2cm}|P{2cm}|}
\cline{2-7}    \multicolumn{1}{c|}{} & \multicolumn{3}{c|}{All isoforms} & \multicolumn{3}{c|}{Subset isoforms} \bigstrut\\
\hline
Method & SCC   & NRMSE & MRD   & SCC   & NRMSE & MRD \bigstrut\\
\hline
Deg. EM (exact) & 0.832 & 0.455 & \textbf{0.006} & \textbf{0.793} & 0.711 & \textbf{0.162} \bigstrut\\
\hline
Deg. EM (emp.) & \textbf{0.852} & \textbf{0.416} & 0.016 & 0.783 & \textbf{0.446} & 0.277 \bigstrut\\
\hline
\end{tabular}%
}
\caption[Summary of metrics across simulated datasets for model variations]{Summary of metrics across simulated datasets for model variations. We report the mean SCC, NRMSE and MRD across the five datasets for all isoforms and subset isoforms separately. Bold values indicate the best performance for each column.}
\label{tab:summary-1}
\end{table}%

Next, we examine the performance on each dataset separately by examining performance on each metric across the datasets for all and subset isoforms (Fig. \ref{fig:4-1-scc-nrmse}). We observe an interesting trend in both the SCC and NRMSE: as the degradation rate increases, the empirical model performs slightly better compared to the exact model. In contrast, the exact model dominates the empirical model in terms of the MRD, especially for subset isoforms.  

To investigate this further, we visualised estimates and fitted a kernel density estimate on the subset isoforms only for datasets with degradation rates 0.2 and 0.5 (Fig. \ref{fig:4-1-scatter}). On a dataset with relatively lower degradation rate (0.2), the exact model performs qualitatively well, with the density of points over the diagonal (Fig. \ref{fig:4-1-scatter}a), while the empirical model underestimates counts for certain isoforms (Fig. \ref{fig:4-1-scatter}b). However, for higher degradation rates (0.5), the exact model appears to aggressively overestimate counts for a subset of isoforms (Fig. \ref{fig:4-1-scatter}c) compared to the empirical model, where the counts are moderately distributed about the diagonal (Fig. \ref{fig:4-1-scatter}d). This explains the significant increase in NRMSE in the exact model for larger degradation rates compared to the empirical model (Fig. \ref{fig:4-1-scc-nrmse}) because the NRMSE penalizes large errors more strongly than moderate errors. 

Based on our analyses in section \ref{sec:deg-est}, we observed that degradation rates of 0.1-0.2 from direct RNA-seq data. In this range, the two variations perform comparably, and have a high SCC with each other ($\mathbb{E}[d]$=0.1, SCC=0.957, $\mathbb{E}[d]$=0.2, SCC=0.909).  

\begin{figure}[H]
    \centering
    \includegraphics[width=0.9\textwidth]{figures/sec-4-1-scc-nrmse.png}
    \caption[SCC, NRMSE and MRD across simulated datasets for model variations]{SCC, NRMSE and MRD across simulated datasets for model variations. Here, each point is a dataset with constant expected degradation $\mathbb{E}[d]=\{0.05,0.1,0.2,0.4,0.5\}$. \textbf{a.} SCC on all isoforms. \textbf{b.} SCC on subset isoforms. \textbf{c.} NRMSE on all isoforms. \textbf{d.} NRMSE on subset isoforms. \textbf{f.} MRD on all isoforms. \textbf{e.} MRD on subset isoforms.}
    \label{fig:4-1-scc-nrmse}
\end{figure}

\begin{figure}[H]
    \centering
    \includegraphics[width=\textwidth]{figures/sec-4-1-scatter-hard.png}
    \caption[Scatter plots across simulated datasets for model variations]{Scatter plots of the simulated and estimated counts log2 scale across simulated datasets with $\mathbb{E}[d]=0.2$ and $\mathbb{E}[d]=0.5$ for the exact and empirical model. \textbf{a.} Exact model with degradation 0.2. \textbf{b.} Empirical model with degradation 0.2. \textbf{c.} Exact model with degradation 0.5. \textbf{d.} Empirical model with degradation 0.5.}
    \label{fig:4-1-scatter}
\end{figure}

\subsection{Comparisons with existing methods}

We now compare our model against Bambu with and without bias modeling, FLAIR and NanoCount on the simulated datasets, comparing the estimates returned by each method with the simulated ground truth. We analyse counts for all isoforms that we included in the simulation and where the sum of counts across the methods is greater than zero. All methods perform reasonably well on the simulated data across all isoforms, achieving SCC$>$0.69 across all five simulated datasets (Table \ref{tab:summary-2}). NRMSE across the methods are comparable, but MRD for the other methods are about one order of magnitude larger than those attained by our model. However, we observe stark differences between the different methods on the subset isoforms. In particular, methods that do not model and correct for bias (Bambu (no bias), FLAIR, NanoCount) perform considerably poorer than methods that do.   

\begin{table}[htbp]
\centering
\resizebox{\columnwidth}{!}{%
\begin{tabular}{|l|P{2cm}|P{2cm}|P{2cm}|P{2cm}|P{2cm}|P{2cm}|}
\cline{2-7}    \multicolumn{1}{r|}{} & \multicolumn{3}{c|}{All isoforms} & \multicolumn{3}{c|}{Subset isoforms} \bigstrut\\
\hline
Method & SCC   & NRMSE & MRD   & SCC   & NRMSE & MRD \bigstrut\\
\hline
Deg. EM (exact) & 0.845 & 0.449 & \textbf{0.002} & \textbf{0.815} & 0.699 & \textbf{0.135} \bigstrut\\
\hline
Deg. EM (emp.) & \textbf{0.863} & \textbf{0.41} & 0.012 & 0.801 & \textbf{0.439} & 0.241 \bigstrut\\
\hline
Bambu & 0.771 & 0.599 & 0.101 & 0.671 & 1.009 & 0.36 \bigstrut\\
\hline
Bambu (no bias) & 0.739 & 0.637 & 0.107 & 0.432 & 1.165 & 0.926 \bigstrut\\
\hline
FLAIR & 0.696 & 0.697 & 0.081 & 0.153 & 1.176 & 0.974 \bigstrut\\
\hline
NanoCount & 0.759 & 0.566 & 0.105 & 0.408 & 1.056 & 0.931 \bigstrut\\
\hline
\end{tabular}%
}
\caption[Summary of metrics across simulated datasets for different methods]{Summary of metrics across simulated datasets for different methods. We report the mean SCC, NRMSE and MRD across the five datasets for all isoforms and subset isoforms separately. Bold values indicate the best performance for each column.}
\label{tab:summary-2}
\end{table}%

Next, we examine the performance on each dataset separately for all the methods on all and subset isoforms separately (Fig. \ref{fig:4-2-scc-nrmse}). Both Deg. EM (exact) and Deg. EM (empirical) improve the performance on SCC compared to all methods on the simulated data (Fig \ref{fig:4-2-scc-nrmse}a,b). Notably, Bambu with bias modeling improves the performance over the whole range of degradation rates compared to Bambu without bias modeling. We note also that Bambu's performance drops quickly as the degradation rate increases, as its degradation rate is calibrated to around 0.1-0.2.   

\newpage

\begin{figure}[H]
    \centering
    \includegraphics[width=0.9\textwidth]{figures/sec-4-2-scc-nrmse.png}
    \caption[SCC, NRMSE and MRD across simulated datasets for different methods]{SCC, NRMSE and MRD across simulated datasets for different methods. Here, each point is a dataset with constant expected degradation $\mathbb{E}[d]=\{0.05,0.1,0.2,0.4,0.5\}$. \textbf{a.} SCC on all isoforms. \textbf{b.} SCC on subset isoforms. \textbf{c.} NRMSE on all isoforms. \textbf{d.} NRMSE on subset isoforms. \textbf{f.} MRD on all isoforms. \textbf{e.} MRD on subset isoforms.}
    \label{fig:4-2-scc-nrmse}
\end{figure}

\newpage

\begin{figure}[H]
    \centering
    \includegraphics[width=0.9\textwidth]{figures/sec-4-2-scatter-hard.png}
    \caption[Scatter plots across simulated datasets for different methods]{Scatter plots of the simulated and estimated counts log2 scale across simulated datasets with $\mathbb{E}[d]=0.2$ for all methods. \textbf{a.} Deg. EM (exact) model. \textbf{b.} Deg. EM (emp.) model. \textbf{c.} Bambu with bias modeling. \textbf{d.} Bambu with no bias modeling. \textbf{e.} FLAIR. \textbf{f.} NanoCount.}
    \label{fig:4-2-scatter}
\end{figure}

\newpage

\subsection{Runtime analysis}

We measured the end-to-end runtime taken for all methods (Fig. \ref{fig:runtime}) excluding the read alignment step. Bambu and FLAIR were run with 12 threads. NanoCount and the current verison of our model do not implement parallelisation. Across the five datasets, NanoCount was the fastest, followed by Deg. EM (exact). Bambu and FLAIR performed similarly, though FLAIR had a much larger variance in the runtime. Even though the time complexity for both Deg. EM (exact) and Deg. EM (emp.) is $\mathcal{O}(N_A)$, where $N_A$ is the number of alignments, Deg. EM (emp.) is much slower than all other methods because calculation of the empirical read length-isoform agreement probabilities is computationally intensive. Even though the runtimes for the empirical model are longer, they are still tolerable in absolute terms, and comes with improvements in accuracy for accurately quantifying degraded reads. Nevertheless, this is one area for improvement for future iterations of our model.  

\begin{figure}[H]
    \centering
    \includegraphics[width=0.8\textwidth]{figures/sec-4-runtime.png}
    \caption[Runtime across simulated datasets for different methods]{Runtime (mins) across simulated datasets for different methods. Times are log10 transformed.}
    \label{fig:runtime}
\end{figure}

\section{Evaluations on real data}

Six H9 samples (600ng mRNA + 1\% spike-in of 6ng SIRV-4) sequenced with direct RNA-seq.

\subsection{Comparisons with existing methods}

\chapter{Conclusion}

\section{Summary}

In this thesis, we have characterised bias in 


\section{Further work}

\subsection{Gene-specific degradation}

\subsection{Read position-isoform agreement}

\subsection{Additional bias in short isoforms}




\appendix
\titleformat{\chapter}[display]
  {\normalfont\huge\bfseries}% <- font for label "Appendix A", default \huge
  {\chaptertitlename\ \thechapter}
  {20pt}
  {\Large}% <- font for title, default \Huge

\chapter{Generating novel isoform models}\label{ap:gen-novl-iso}

In this appendix, we describe our approach for generating novel gene isoform models. We first select a set of seed reference isoforms and introduce modifications that are well known to occur via alternative isoform regulation \textit{in vivo}. These modifications include the use of alternative 5' start sites, alternative 3' end sites, alternative splice donors and acceptors, exon skipping, intron retention, and the introduction of new exons and introns. In addition, we also introduce a new modification, termed \textit{subset} isoform, that drops exons from the reference isoform from the 5' end. Introduction of these novel subset isoforms increases the number of multi-mapping reads, making the process of assigning these reads to the correct isoform more complex. 

%<Insert figure here describing novel isoform models>

Because the \texttt{minimap2} aligner uses splice site signals to map reads, we ensure that the generated novel isoform models conform to these splice site signals. We do so by performing splice site correction whenever necessary, i.e., for the alternative splice donor, alternative splice acceptor, new exon and new intron modifications. 



Software for generating novel isoform models was written in \texttt{python} and can be found at \url{https://github.com/jleechung/noviso}.

\chapter{Simulating degraded reads}\label{ap:sim-deg-reads}

In this appendix, we describe our approach for simulating reads based on constant expected degradation. We select protein-coding isoforms and processed transcripts from GRCh38 annotation to simulate reads for. The distribution of counts follows a negative binomial distribution (parameterised here by the mean and rate), which is commonly used along with the Poisson distribution for modeling count data in RNA-seq. The negative binomial distribution has an added advantage of modeling overdispersion in the data, where the variance is greater than the mean. We note that the negative binomial distribution is equivalent to a mixture of Poisson distributions where the rate parameter is distributed according to a gamma distribution. 

To generate reads for a transcript of length $\mathrm{len}(j)$ given a degradation rate $d$, we first compute the maximum read length $\ell_\mathrm{max}=1/d$. The probability of generating a degraded read is $p_d=\min(\mathrm{len}(j)/\ell_\mathrm{max}, 1)$ and the probability of generating a full-length read is $1-p_d$. We now generate a read length with the following algorithm:
\begin{enumerate}
    \item Generate $U\sim Unif(0,1)$.
    \item If $U<p_d$, generate read length $\ell_i\sim Unif(0,\min(\mathrm{len}(j),\ell_\mathrm{max}))$.
    \item Otherwise, generate read length $\ell_i=\mathrm{len}(j)$
\end{enumerate}
For constant degradation, degraded reads follow a uniform distribution upper bounded by the length of the transcript isoform or the maximum read length, whichever is smaller. Once the read length $\ell_i$ is generated, we simply slice the transcript isoform sequence from the 3' end such that the resulting sequence is of length $\ell_i$. The reads simulated are thus perfect reads with 0\% error rate and no indels. 

Software for generating novel isoform models was written in \texttt{python} and can be found at \url{https://github.com/jleechung/shamread}.

\chapter{Proof of concavity of log-likelihood function}\label{sec:proof-log-lik}

Hello world.

\chapter{Derivations for variational distributions}\label{sec:variational-dist}

\begin{equation}
\begin{split}
    \log q(\bm{Z}) & = \mathbb{E}_{\bm\theta}\left[\log p(\bm{R},\bm{Z},\bm{\theta})\right] + \textrm{const.} \\
    & = \mathbb{E}_{\bm\theta}\left[\log p(\bm{R}\mid\bm{Z})\right] + \mathbb{E}_{\bm\theta}\left[\log p(\bm{Z}\mid\bm{\theta})\right] + \textrm{const.} \\
    & = \sum_i\sum_j z_{ij}\log\left[a_{ij}\cdot p(\ell_i\mid d_j,z_{ij}=1)\right] + \sum_i\sum_j z_{ij}\mathbb{E}_{\bm\theta}\left[\log\theta_j\right] + \textrm{const.}
\end{split}
\end{equation}

\chapter{Evaluation metrics}

\lipsum[32]

\printbibliography[heading=bibintoc]

\end{document}
