Third generation sequencing technologies have enabled the production of long reads ranging from tens to thousands of bases in length \cite{Pollard2018}, and have shown promise in resolving many challenges in genomics and transcriptomics \cite{Bolisetty2015, Byrne2017, DeCoster2019, Liu2019, Mantere2019, Nurk2021}. In particular, long-read technologies enable greater insight into the transcriptome and its complexity, which is crucial in understanding the functioning of cells and their biological processes. These technologies allow accurate detection of full-length transcript isoforms and novel splice junctions while mitigating biases associated with short-read technologies, enabling more accurate quantification of reference and novel isoforms. Nevertheless, biases are still present in long-read technologies, albeit to a lesser extent. 

This project focuses on a particular bias present in long-read \gls{rnaseq} referred to as \textit{degradation bias}. This bias arises due to the fact that from the time a transcript is synthesized to when it is sequenced, it is subject to multiple factors that results in its degradation, primarily from the 5' end of the transcript. Consequently, the observed reads are often truncated and \textit{degraded}, potentially resulting in ambiguity in read assignment to transcript isoforms. This, in turn, leads to erroneous quantification estimates. As degradation bias in long-read \gls{rnaseq} has not been extensively covered in existing literature, we attempt to do so here by characterising the bias and its effects on quantification from long-read \gls{rnaseq}, and developing a framework to model and correct such bias by producing \textit{degradation-aware quantification estimates}. We incorporate this model into bambu, a tool for context-aware transcript discovery and quantification from long-read \gls{rnaseq}. 

To evaluate quantification estimates from \gls{rnaseq}, experimental datasets with known ground truth or simulated datasets are typically used, and estimates are evaluated based on their concordance with the ground truth. Here, we generate simulated long-read \gls{rnaseq} data as experimental datasets with known ground truth are unavailable. In order to test our bias models, we develop an approach to generating novel transcript models that is capable of generating adversarial transcript models where not correcting for degradation bias can lead to significantly erroneous quantification estimates. For evaluating degradation-aware quantification estimates, we employ several reference-based and reference-free metrics that measure the concordance of estimates with the ground truth along with the variance of estimates acrosss technical replicates.   

\section{Review}

Here, we review various concepts and existing literature relevant to our aim of modeling bias in long-read \gls{rnaseq}. We first examine (i) biases in short-read \gls{rnaseq} technologies and how they are accounted for by existing methods, which provide some ideas on how to handle biases in long-read \gls{rnaseq}. Next, we review (ii) long-read \gls{rnaseq} technologies and how they mitigate biases in (i), and (iii) biases that long-read technologies themselves possess. 

